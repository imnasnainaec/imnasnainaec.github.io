\documentclass[11pt]{amsart}
\usepackage{amsmath}

\newtheorem{thm}{Theorem}

\pagestyle{empty}
%Style section
\setlength{\textheight}{9in}
\setlength{\textwidth}{6.85in}
\topmargin=-1in
\headheight=0in
\headsep=.5in
\hoffset  -1in
%\hoffset=-1in
% \pagestyle{plain}
% \setlength{\topmargin}{0in}
% \setlength{\headsep}{0in}
% \setlength{\oddsidemargin}{0in}
% \setlength{\evensidemargin}{0in}

\begin{document}

\thispagestyle{empty} %use this to get rid of page numbers

\noindent Math 170-007 - Finite Mathematics \hfill Exam \#1 - 2013 February 11\\\
\noindent Instructor: Danny Rorabaugh

\vfill

$$\vdots$$

\vfill

\noindent 2. The demand for explosive gum is 500 sticks each week if the British government is giving them to agencies free of charge, and drops to 400 each week if the charge is \$100 per stick. However, the government is prepared to supply only 20 sticks per week free of charge, but will supply 1400 each week at \$200 per stick.

[a] Find the associated linear demand and supply functions in terms of the price $p$ in dollars per stick.

[b] At what price (to the nearest cent) should the government charge its agencies for a stick of explosive gum so that there is neither a surplus nor a shortage?\\

\vfill

$$\vdots$$

\vfill

\noindent 4. The best agency in Great Britain, MI6, succeeded in a total of 12 missions yesterday. Some of these missions were in Eurasia, and the others were in the Americas. According to standard protocol, each mission in Eurasia requires 2 agents, whereas each mission in the Americas requires 3 agents. The total number of agents used yesterday was 29.

[a] Write a system of two linear equations for this situation. Clearly defined your two variables

[b] Use any method to answer: how many missions succeeded in each supercontinent yesterday?

\vfill

$$\vdots$$

\vfill

\end{document}
